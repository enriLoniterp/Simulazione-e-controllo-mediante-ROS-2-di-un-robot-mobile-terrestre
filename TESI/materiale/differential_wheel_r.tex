\documentclass[11pt]{article}

\begin{document}

DIFFERENTIAL DRIVE KINEMATICS

Il meccanismo dei Robot a a ruote differenziali consiste nell'avere due ruote che in maniera indipendente avanti o indietro.
Variando la velocità delle due ruote il robot ruota intorno a un punto denominato ICC(InstantaneousCenterofCurvature).

Associamo quindi la velocità angolare con quella tangenziale della ruota e otteniamo il sistema:

$\omega\cdot(R+(l/2))$
$\omega\cdot(R-(l/2))$

Risolviamo il sistema e otteniamo R e $\omega$

$ R = (l/2)\cdot(\frac{Vr+Vl}{Vr-Vl})$
$\omega=\frac{Vr-Vl}{l}$

Ci sono 3 casi speciali:

1)Vl=vR per la quale abbiamo la velocità angolare a 0 e quindi un movimento in linea retta del robot
2) Vl=-Vr con R=0 e quindi una rotazione circolare del robot rispetto al punto medio dell'asse del robot
3) se Vl=0 allora la rotazione è della ruota destra rispetto alla sinistra  viceversa.

Scriviamo la legge della cinematica per Robot Differenziali:
Assumiamo il punto medio del robot alle coordinate (x,y) e consideriamo il sistema di riferimento x e y con \theta l'angolo che si forma tra l'asse delle x e la traiettoria del robot allora possiamo ricavarci il punto ICC come 

$ ICC = [ x - Rsin(\theta), y + Rcos(\theta) ] $

e ad un tempo t + omega la posizione del robot sarà:

http://www.cs.columbia.edu/~allen/F15/NOTES/icckinematics.pdf

(prendere l'immagine dal file).




\end{document}